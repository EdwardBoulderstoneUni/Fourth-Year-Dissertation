\documentclass{entcs} 
\usepackage{CSC8498macro}

%% This document describes the formatting instructions for the CSC8498 final report.

\makeatletter

\def\lastname{Boulderstone}
\begin{document}

\begin{frontmatter}
\title{Rollback Netcode, Implmentation and Adoption}
\author{Edward Boulderstone}
  \address{School of Computing Science, Newcastle University, UK} 
\thanks[nigellemail]{Email:
    \href{mailto:E.Boulderstone@ncl..ac.uk} {\texttt{\normalshape
        E.Boulderstone@ncl..ac.uk}}}

			
				
\begin{abstract} 
Write your abstract here. This should be a concise summary of what your project has been about and what you aim to show. 9pt text here.
\end{abstract}

\begin{keyword}
list the main concepts used in your project, e.g.\ the research area and techniques used.
\end{keyword}
\end{frontmatter}

\section{Introduction}\label{sec: introduction}

This document gives a template for producing your final CSC8498 report using MS Word or LaTeX \cite{thomas2016} additional guidance is given on the website of Elsevier’s Electronic Notes in Theoretical Computer Science \cite{entcs}. LaTeX users should use the modified ENTCS class and style files provided on Blackboard. There may be very slight differences between the Word and LaTeX layouts, although these will make little difference in practice. Do not change the page layout. Section headers numbered and are 14pt bold with 12pt spacing above and 10t spacing below the heading. Body text is 11pt and fully justified (text aligned to both margins). Left margin 4.1cm, right margin 3.5cm, top margin 3cm and bottom margin 5cm. The page header should include module code and page footer includes page numbers only. The first paragraph of each section is not indented; subsequent paragraphs are indented by 0.5cm. Line spacing for general text is 1.15 and there is no additional spacing between paragraphs within a section. 

The introduction should set the scene for your projects by giving a clear project statement and introducing the key concepts. You may also include the background literature here or in a separate section. Do not leave section headers hanging at the bottom of a page; hence the next section starts on a new page.

\section{What you did and how}

It is highly unlikely that you will call this section by this name, it is more likely to be “Design and implementation”, “Model”, “Simulation”, “Experimental design” or such like, depending on your project. It will describe what you did and how you did it! Again, you might want to split this section into multiple sections, or introduce subsections, to separate distinct activities which you undertook, e.g. you could separate “Design” and “Implementation” into different sections. 

\section{Evaluation}

This section might alternatively be called “Experimental results”.  It describes how you assessed your project and what you found out. Where you give results in tables or graphs, remember to highlight in the text the key points that the data shows and, if possible, try to explain why you got any unexpected results.

\section{Conclusions and further work}

In this section you summarise what you did and discovered and (importantly) what else you would have done (or done differently) if you had the chance. It is a chance to reflect on your success.

\section{Acknowledgements (optional and unmarked)}

If anyone else helped you, e.g. by providing code or data, then you should acknowledge their contribution here. Avoid long tributes or anything too personal and try to stick to acknowledgements which are specific to the project. You do not have to acknowledge your supervisor or second marker; they are just doing their jobs and acknowledging them or not will not affect your mark. 

In the references below you will give all the references (in 9 pt text) for papers you have cited in the text above. Ensure that all references are complete and consistently formatted. Make sure that web pages include the date you last accessed them. In general it is good practice to use a bibliography support tool such as Endnote (for Word) or BibTeX (for LaTeX) to compile your references, but it is not essential. % I have not used any tool here.

\begin{thebibliography}{25}


\bibitem{thomas2016} Thomas, N., Formatting guidance for CSC8498 final project report, School of Computer Science, Newcastle University, 2016.

\bibitem{entcs} {\texttt http://www.entcs.org}, last accessed 01/03/2016.


\end{thebibliography}
\end{document}
